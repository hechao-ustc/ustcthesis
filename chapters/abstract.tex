% !TeX root = ../main.tex

\ustcsetup{
  keywords  = {数据湖, Iceberg, 实时湖仓, 小文件合并},
  keywords* = {dissertation, abstract, keywords},
}

\begin{abstract}
  近年来随着大数据、机器学总习、5G等技术飞速发展,数据规模越来越大,数据量呈几何增长,数据来源和类型更加多元化。
  同时随着企业对数据驱动业务需求的深入,也随着海量数据分析技术的成熟,数据仓库已是企业内部进行数据洞察的标准服务。
  但是传统基于Hadoop技术的数据仓库从数据导入到数据分析每个环节都有较大的延迟,使得数据分析的时效性大大变低;
  同时对于数据分析场景的拓展,传统单一的数仓架构也没法满足多变的数据分析需求。为此业界也在探索新一代更为通用的实时数仓/数据湖架构。

  经过对企业大数据发展现状的深入调研,充分了解企业需求和业务挑战,同时结合前沿开源技术发展现状,设计
  并实现了基于iceberg的数据湖分析(Data Lake Analytics,简称DLA)系统。首先对企业内部的多种数据源进行分析总结,然后
  我们根据Apache Iceberg支持多种数据源、多种计算引擎、多种存储介质以及其它优秀特性,设计并实现了数据源管理、元数据管理、数据入湖、
  数据探索四大模块,其中数据源是入湖任务的前提要求,通过注册相应数据源,可在创建入湖任务时选择对应的源表;元数据是描述iceberg表的,
  iceberg表即入湖任务中的目标表;数据入湖是DLA系统的核心流程功能,目的是用户通过该功能将源数据表流程化入湖和查看已申请入湖
  任务执行情况;数据探索模块基于spark和presto实现了数据的查询使用功能。其中在元数据管理中,我们提供了iceberg表的自动优化
  功能,包括合并小文件、清理过期快照数据、删除孤儿文件、生命周期管理等,这项服务使得用户不再需要自己写程序来进行优化,降低了用户的运维成本,
  使用户可以一键启动该服务,并会根据表的若干指标及历史执行情况判断所需资源,并配有告警机制,及时通知专业运维处理。

  目前数据湖分析系统已经上线运行,为企业用户提供了稳定可靠的一站式数据入湖服务,并且对接企业内部的大数据计算平台、数据查询平台
  报表平台等多个下游业务系统,满足了企业数据开发需求,创造了巨大的价价值。

\end{abstract}

\begin{abstract*}
  This is a sample document of USTC thesis \LaTeX{} template for bachelor,
  master and doctor. The template is created by zepinglee and seisman, which
  orignate from the template created by ywg. The template meets the
  equirements of USTC thesis writing standards.

  This document will show the usage of basic commands provided by \LaTeX{} and
  some features provided by the template. For more information, please refer to
  the template document ustcthesis.pdf.
\end{abstract*}
