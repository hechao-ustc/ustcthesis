% !TeX root = ../main.tex

\ustcsetup{
  keywords  = {数据湖, Iceberg, 实时湖仓, 自动优化},
  keywords* = {Data Lake, Iceberg, Real-time Lakehouse, Auto Optimize},
}

\begin{abstract}

近年来随着大数据、机器学习和5G等技术的飞速发展,数据规模不断增大,数据来源和类型也变得更加多元化。
同时随着企业对数据驱动业务需求的不断深入,也随着海量数据分析技术的成熟,数据仓库已成为企业内部数据洞察的标准服务。然而,
传统基于Hadoop技术的数据仓库从数据导入到数据分析每个环节都有较大的延迟,使得数据分析的时效性大大降低。
同时,对于数据分析场景的拓展,传统单一的数仓架构也无法满足多变的数据分析需求。
为此,业界也在探索新一代更为通用的实时数仓和数据湖架构。

我们对企业大数据发展现状进行了深入调研,并且充分了解了企业所面临的需求和疼点,同时结合前沿开源技术发展现状,设计
并实现了基于Iceberg的数据湖分析系统(Data Lake Analytics,简称DLA)。
我们首先对企业内部的多种数据源进行分析总结,然后
根据Apache Iceberg支持多种数据源、多种计算引擎、多种存储介质以及其它优秀特性,
设计并实现了数据源管理、元数据管理、数据入湖、
数据探索四大模块。其中数据源是入湖任务的前提要求,通过注册相应数据源,用户可在创建
入湖任务时选择对应的源表;元数据是描述Iceberg表的,Iceberg表即入湖任务中的
目标表;数据入湖是DLA的核心流程功能,目的是用户通过该功能将源数据表流程化入湖
和查看已申请入湖任务执行情况;数据探索模块基于Spark和Presto实现了数据的查询和分析使用功能。

在元数据管理模块中,我们提供了Iceberg表的自动优化服务,包括合并小文件、清理过期快照数据、
删除孤儿文件、生命周期管理等,这项服务使得用户不再需要自己写程序来进行运维,降低了用户的
运维成本,用户可以一键启动该服务,并会根据表的若干指标及历史执行情况判断所需资源,并配有
告警机制,及时通知专业运维处理。

目前,数据湖分析系统已经上线运行,为企业用户提供了稳定可靠的一站式数据入湖服务,并且对接
企业内部的大数据计算平台、数据查询平台和报表平台等多个下游业务系统,满足了企业数据开发需求
,创造了巨大的价值。

\end{abstract}

\begin{abstract*}

In recent years, with the rapid development of big data, machine learning,
5G, and other technologies, the scale of data has been continuously
increasing, and the sources and types of data have become more diverse.
At the same time, as the demand for data-driven business in enterprises
continues to deepen and the maturity of massive data analysis technology,
data warehouse has become a standard service for internal data insights in
enterprises. However, the traditional Hadoop-based data warehouse has a
significant delay in every link from data import to data analysis, greatly
reducing the timeliness of data analysis. At the same time, for the expansion
of data analysis scenarios, the traditional single data warehouse architecture
cannot meet the changing data analysis needs. Therefore, the industry is also
exploring a new generation of more universal Real-time Lakehouse and Data Lake architectures.

We conducted in-depth research on the development status of enterprise big
data and fully understood the needs and pain points faced by enterprises,
and combining the development status of cutting-edge open-source technology,
we designed and implemented a Data Lake Analytics System (DLA)
based on Iceberg. We first analyzed and summarized various
data sources within the enterprise, and then designed and implemented four
modules of data source management, metadata management, data import, and data
exploration based on the various data sources, multiple computing engines,
multiple storage media, and other excellent features supported by Apache Iceberg.
Among them, data source is a prerequisite for the import task. By registering
the corresponding data source, users can select the corresponding source table
when creating an import task. Metadata describes the Iceberg table, which is the
target table in the import task. Data import is the core process function of DLA.
The purpose is for users to streamline the import of source data tables and view
the execution status of applied import tasks. The data exploration module implements
data query and analysis functions based on Spark and Presto.

In the metadata management module, we provide an auto optimize service
for Iceberg tables, including merging small files, cleaning up expired snapshot,
deleting orphan files, and lifecycle management. This service eliminates
the need for users to write programs for operations and maintenance, reduces users'
operation and maintenance costs, and allows users to start the service with one
click. It will judge the required resources based on several indicators and
historical execution results of the table and is equipped with an alarm mechanism
to notify professional operations and maintenance for timely processing.

Currently, the Data Lake Analysis System has been launched and operated,
providing stable and reliable one-stop data import services for enterprise
users. It also connects with multiple downstream business systems such as
the enterprise's big data computing platform, data query platform, and report
platform to meet enterprise data development needs and create significant value.

\end{abstract*}
