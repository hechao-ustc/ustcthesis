% !TeX root = ../main.tex

\chapter{绪论}

本章节主要对论文研究的项目——基于Iceberg的数据湖分析系统的设计与
实现,进行概要介绍。从课题背景和意义、国内外研究现状进行分析,
确定了本论文的研究目标,并细化了研究内容。最终概述了本论文的组织结构。

\section{课题背景和意义}

如今,随着互联网的发展和层出不穷的各种应用,互联网产生着大量的数据,
如何有效存储和处理这些大规模数据成为了一个亟待解决的问题\cite{1}。为此,
数据仓库已是企业内部进行数据洞察的标准服务。但是传统基于Hadoop技术\cite{2}
的数据仓库从数据导入到数据分析每个环节都有较大的延迟,使得数据分析的时效
性大大变低;同时对于数据分析场景的拓展,传统单一的数仓架构也没法满足多变
的数据分析需求。现如今Hadoop生态圈的数据仓库大多面向的是离线场景,数据
从接入直到展示存在着小时级以上的延迟,这其中一方面是因为批处理框架本身延
迟的限制,另一方面也是因为对于变化数据捕捉能力的缺失。

为了解决现有数仓架构的时延问题,业界也在该领域进行了较多的探索,其中最主要的
探索是利用Lambda架构\cite{3}解决海量数据处理的时延问题。Lambda架构是一种常见的
数据处理体系结构,它的数据处理依赖流式计算层(Streaming Layer)和批处理
计算层(Batch Layer)的双重计算。Lambda架构虽然解决了海量数据处理的时延问题,
但是"批流分离"需要维护两条处理链路,这增大了研发的复杂性。为了解决Lambda架构的复杂性,也出现
有了诸多的改进方案,比如Kappa架构\cite{4}。Kappa架构充分利用了
流计算具有天然的分布式特性,因此其扩展性更好。我们可以通过提高流计算的并发性
和增加流式数据的"时间窗口"来将批量处理和流式处理两种计算模式统一起来。
这种架构的简单性使得我们避免了维护两套系统并保持结果一致的问题,同时也很好地解决了
数据订正的问题。但是,它也存在一些问题,比如消息中间件的回放困难和修改缺乏灵活性。
同时OLAP分析性能低下,无法利用到列裁剪、谓词下推、向量化等现代引擎的常用优化手段。

我们采用数据湖技术来解决Lambda架构和Kappa架构引入的问题。
类比于大自然的水,数据湖中汇聚了各个江河湖海的未经加工的数据。
数据湖是一个存储原始格式数据的库或系统,不需要预先对数据进行结构化处理,直接按原样存储\cite{5,6,7,8}。
而本文中的数据湖是基于Apache Iceberg实现的,根据官方的定义,Iceberg是一种表格
式(Table Format)。我们可以简单理解为,它是计算层(如Flink、Spark)和存储层(如Orc\cite{9}、Parquet\cite{10})之间的中间层。
与底层存储格式最大的不同是,它不定义数据的存储方式,而是定义数据和元数据的组织方式,并向上提供了统一的"表"的语义。
我们可以在Hive中创建一个Iceberg格式的表。将数据写入Iceberg中使用Flink或Spark,
然后可以通过其他方式(如Spark、Flink、Presto\cite{11}等)读取该表。
我们基于Iceberg并依托公司内部实时流计算平台和已有的交
互式查询引擎Presto和Spark的能力,打造了下一代的实时湖仓——数据湖分析系统DLA。DLA可以轻
松完成T+0实时入湖,并支持批流融合、秒级分析、事务语义、挖掘和探索数据价值等。

\section{国内外发展现状}

随着数据量的急剧增长和数据类型的多样化,数据湖作为一种新型的数据存储和处理方式,
越来越受到企业和学术界的关注。而作为数据湖的存储格式之一,Iceberg也得到了广泛
的研究和应用。本文将介绍国内外基于Iceberg的数据湖平台的发展现状,并对其优缺点进行分析。

\subsection{国外发展现状}

(1)Uber的数据湖平台

Uber是全球最大的打车服务平台之一,拥有海量的乘客和司机数据\cite{37}。为了更好地管理和利用这些数据,
Uber在其数据平台的升级中,也采用了Iceberg作为数据湖的存储格式。通过使用Iceberg,Uber
能够更好地管理和利用其数据湖中的数据,提高数据处理效率和可靠性。

(2)LinkedIn的DataHub项目

DataHub是LinkedIn开发的一款企业级数据平台,支持数据发现、数据集成、数据管理等功能\cite{38}。
DataHub基于Iceberg存储格式,能够更好地管理和利用LinkedIn的数据湖中的数据,提高数据
处理效率和可靠性。DataHub在LinkedIn内部得到了广泛应用,并已经开源。

(3)Comcast的数据湖平台

Comcast公司是美国最大的有线电视和互联网服务提供商之一,其数据湖平台是基于Iceberg的
设计与实现。该平台主要用于存储和管理各种类型的数据,包括用户数据、设备数据、运营数据等。
通过使用Iceberg格式作为数据存储格式,Comcast可以实现数据的版本控制和数据快照功能,方便对数据进行管理和分析。
Comcast的数据湖平台是其数据驱动战略的重要组成部分,为企业提供了更加灵活、可靠和高效的数据管理和分析解决方案,有助于推动其业务创新和发展。

\subsection{国内发展现状}

(1)腾讯云数据湖计算DLC(Data Lake Compute)

腾讯云DLC平台是基于Iceberg存储格式的一款企业级数据存储和处理平台。在DLC平台中,
Iceberg扮演了重要的角色,被用于数据的存储和管理。DLC平台采用了Iceberg格式进行
数据存储,支持数据版本控制和快速查询。同时,腾讯还开发了自己的元数据管理系统,用于管理和
维护Iceberg表的元数据信息,包括表结构、数据版本、快照等信息。元数据管理系统与Iceberg表
相结合,实现了更加灵活和可靠的数据管理。腾讯云DLC平台除了Iceberg,还包括Flume、Kafka、
Hadoop、Spark等工具,用于数据采集、处理和分析。

(2)阿里巴巴的MaxCompute数据湖平台

MaxCompute是阿里巴巴开发的一款企业级数据处理和分析平台,支持PB级别的数据处理和分析\cite{39}。
MaxCompute基于Iceberg存储格式,能够更好地管理和利用阿里巴巴的数据湖中的数据,提高
数据处理效率和可靠性。MaxCompute已经成为了阿里巴巴内部数据处理和分析的重要工具。

(3)京东的数据湖平台

京东的数据湖平台是基于Iceberg存储格式的一款企业级数据存储和处理平台,支持PB级别的
数据处理和分析。京东的数据湖平台具有更高的灵活性、可扩展性、查询性能和可靠性,支持
复杂的数据类型和多版本数据管理。京东的数据湖平台已经成为了京东内部数据处理和分析的重要工具。

(4)百度数据湖管理与分析平台EasyDAP

EasyDAP采用Iceberg格式作为存储格式,支持Apache Spark、Presto等分析工具。
在EasyDAP中,百度使用了Iceberg的版本管理和快照功能,支持数据版本控制和数据快照。
同时,百度也开发了自己的元数据管理系统,用于管理和维护Iceberg表的元数据信息,包括表
结构、数据版本、快照等信息。通过元数据管理系统,百度实现了更加灵活和可靠的数据管理。
除了Iceberg,百度的EasyDAP平台还包括Hadoop、Spark等工具,用于数据处理和分析。
在数据采集方面,百度也开发了自己的数据采集工具,包括数据采集、清洗、转换等功能。
在数据可视化和应用方面,百度采用自己开发的数据可视化工具,包括数据报表、图表等,方便用户进行数据分析和应用。

总体而言,这些企业应用Iceberg的案例表明,Iceberg作为一种新型的数据湖存储格式,
具有更高的灵活性、可扩展性、查询性能和可靠性。同时,Iceberg的广泛应用也证明了
其在实际应用中的可行性和有效性,为企业提供了更高效、更可靠、更灵活的数据存储和处理方案。

\section{本文的研究目标和内容}

该项目旨在解决以往数仓无法避免的架构延迟、高昂的数据修改成本等业务疼点,以实现实时接入、批流一体化、秒级分析、支持事务语义为目标。
为了实现该目标,本文设计并实现了基于Iceberg的数据湖分析系统,具体内容如下:

\subsection{数据源管理模块}

数据源管理模块主要对数据源进行管理,数据源是入湖任务必须设置的,是Iceberg入湖的源头,从数据源分类上来看,
数据湖分析系统支持关系型数据库源(mysql)以及消息队列数据源(tube、kafka\cite{31,40}、pulsar\cite{17}),该模块支持数据源创建、查看、编辑、删除功能。

\subsection{元数据管理模块}

由于企业业务数据多源异构的特点,难以将数据进行统一存储,所以通过建立统一的元数据模型\cite{18,19},
进行元数据接入,数据本身还在原各自系统中,通过提供跨系统元数据管理能力,提高管理便捷性,减少用户数据搬运行为。
数据湖要保证存储的数据资源语义的一致性,必须具备强大的元数据管理能力,这是进行大数据分析的基本前提\cite{20}。

我们使用的元数据服务是内部部署的Hive Metastore,使用Hive Metastore来存储Iceberg表的元数据,Iceberg表即目标表,
是入湖任务的前提要求,可以创建新表或者关联已有的表,在创建入湖任务时即可选择对应的目标表。元数据管理模块主要对Iceberg
元数据进行管理,包含Iceberg表的创建、查看、修改。

\subsection{入湖任务管理模块}

数据入湖模块是DLA系统核心流程功能,目的是用户通过该功能将源数据表流程化入湖和查看已申请入湖任务执行情况。
入湖任务分为三类,分别为实时数据入湖、存量数据入湖和关系型数据入湖,三类入湖任务的创建都需要填写基础信息、源表、
目标表、参数及资源。

\subsection{数据探索模块}

数据探索功能是数据入湖后,用户需要进行数据查看或者数据分析时使用的工具,目前主要依赖内部查询平台实现数据探索功能。
我们的数据湖分析平台和查询平台之间是通过Hive Metastore\cite{30}进行打通的,在查询平台上,用户可以编写sql进行数据查询,
底层支持的计算引擎有presto\cite{16}和spark\cite{23}。当然查询平台也提供了api接口,通过这些
接口可以满足各种下游业务需求,如实时查询、BI报表、机器学习、交互式查询等。

\subsection{自动优化服务}

自动优化服务的目的是降低用户的运维成本,使用户可以一键启动该服务,不需要自己写程序来优化,并会根据表的若干指标及历史执行情况
判断所需资源,并配有告警机制,及时通知专业运维处理。目前,自动优化服务包括:合并小文件、清理过期快照数据、删除孤儿文件、生命周期管理。
其中,合并小文件是为了减少小文件过多导致数据查询慢的问题,对于小文件合并我们在Iceberg侧做了很大的优化,实现了计算资源的合理使用;
清理过期快照数据是清理commit时间过期的快照(snapshot),过期时间可以设置,以小时为粒度;
删除孤儿文件是清除因为commit冲突等原因产生的孤儿文件;
生命周期管理是对数据进行生命周期管理,及时清除掉不需要的数据,提高存储空间的使用。

\section{本文的组织结构}

第一章——绪论。介绍项目背景及意义,分析数据湖分析系统在国内外的发展现状,明确本研究的主要内容,概述本人的主要工作和论文结构。

第二章——系统相关技术介绍。本章旨在介绍系统设计与实现中所涉及的主要技术,如Iceberg、Flink、SpringBoot、MySQL等。
同时也会简要介绍这些技术的相关内容。

第三章——系统需求分析。本章节首先概述了系统需求架构,然后详细分析了系统的功能性需求以确定解决问题的方案,
并最终总结了系统的非功能性需求分析结果。

第四章——系统概要设计。本章重点介绍系统的概要设计和详细设计,其中包括系统总体功能模块、技术实现架构以及数据库详细设计。

第五章——系统详细设计与实现。本章在第四章系统概要设计的基础上,详细介绍了数据源管理模块、元数据管理模块、
入湖任务管理模块、数据探索模块以及自动优化服务的实现。

第六章——系统测试。本章为整个系统设计了测试用例,包括功能性测试、非功能性测试和性能测试。同时展示了部分测试结果,并得出了测试结论。

第七章——结论与展望。本章对系统开发工作进行了归纳和总结,并根据这些总结对本系统未来的工作进行了展望。

\section{本章小结}

本章首先介绍了基于Iceberg的数据湖分析系统的背景、意义以及国内外的研究现状,
然后确定了项目的研究目标并阐述了研究内容,最终对论文的组织结构进行了说明。
