% !TeX root = ../main.tex

\chapter{绪论}

本章节主要对论文研究的项目——基于Iceberg的数据湖分析系统的设计与
实现,进行概要介绍。分别从课题背景和意义、国内外研究现状进行分析,制定了
本论文的研究目标,细化论文的研究内容,最后概述本论文的组织结构。

\section{课题背景和意义}

如今,随着互联网的发展和层出不穷的各种应用,互联网产生着大量的数据,
如何有效存储和处理这些大规模数据成为了一个亟待解决的问题[1]。为此,
数据仓库已是企业内部进行数据洞察的标准服务。但是传统基于Hadoop技术[2]
的数据仓库从数据导入到数据分析每个环节都有较大的延迟,使得数据分析的时效
性大大变低;同时对于数据分析场景的拓展,传统单一的数仓架构也没法满足多变
的数据分析需求。现如今Hadoop生态圈的数据仓库大多面向的是离线场景,数据
从接入直到展示存在着小时级以上的延迟,这其中一方面是因为批处理框架本身延
迟的限制,另一方面也是因为对于变化数据捕捉能力的缺失。

为了解决现有数仓架构的时延问题,业界也在该领域进行了较多的探索,其中最主要的
探索是利用Lambda架构[3]解决海量数据处理的时延问题。Lambda架构是一种常见的
数据处理体系结构,它的数据处理依赖流式计算层(Streaming Layer)和批处理
计算层(Batch Layer)的双重计算。Lambda架构虽然解决了海量数据处理的时延问题,
但是“流批分离”的处理链路增大了研发的复杂性。为了解决Lambda架构的复杂性,也出现
有了诸多的改进方案,比如Kappa架构[4]。Kappa架构充分利用了流计算天然的分布式特征,
注定了他的扩展性更好。通过加大流计算的并发性,加大流式数据的“时间窗口”,来统一批处理
与流式处理两种计算模式。这样的架构简单,避免了维护两套系统还需要保持结果一致的问题,
也很好解决了数据订正问题,但它也有它的问题:消息中间件回放困难,修改缺乏灵活性,同时
OLAP分析性能低下,无法利用到列裁剪、谓词下推、向量化等现代引擎的常用优化手段。

为了解决上述Lambda架构和Kappa架构引入的问题,我们引入了数据湖技术。如果把数据比作
大自然的水,那么各个江川河流的水未经加工,源源不断地汇聚到数据湖中[5,6,7,8]。数据湖
是一个以原始格式存储数据的存储库或系统。它按原样存储数据,而无需事先对数据进行结构化处
理。而本文中的数据湖是基于Apache Iceberg实现的,根据官方的定义,Iceberg是一种表格
式(table format)。我们可以简单理解为它是基于计算层(flink、spark)和存储层
(orc[9]、parquet[10])的一个中间层,它与底层的存储格式最大的区别是,它并不定义数据
存储方式,而是定义了数据、元数据的组织方式,向上提供统一的“表”的语义。在hive建立一个
Iceberg格式的表。用flink或者spark写入iceberg,然后再通过其他方式来读取这个表,比如
spark、flink、presto等[16]。我们基于Iceberg并依托公司内部实时流计算平台和已有的交
互式查询引擎Presto和Spark的能力,打造了下一代的实时数仓-数据湖分析系统DLA。DLA可以轻
松完成T+0实时入湖,并支持批流融合、秒级分析、事务语义、挖掘和探索数据价值等。

\section{国内外发展现状}

随着数据量的急剧增长和数据类型的多样化,数据湖作为一种新型的数据存储和处理方式,
越来越受到企业和学术界的关注。而作为数据湖的存储格式之一,Iceberg也得到了广泛
的研究和应用。本文将介绍国内外基于Iceberg的数据湖平台的发展现状,并对其优缺点进行分析。

\subsection{国外发展现状}

(1)Uber的数据湖平台

Uber是全球最大的打车服务平台之一,拥有海量的乘客和司机数据。为了更好地管理和利用这些数据,
Uber在其数据平台的升级中,也采用了Iceberg作为数据湖的存储格式。通过使用Iceberg,Uber
能够更好地管理和利用其数据湖中的数据,提高数据处理效率和可靠性。

(2)LinkedIn的DataHub项目

DataHub是LinkedIn开发的一款企业级数据平台,支持数据发现、数据集成、数据管理等功能。
DataHub基于Iceberg存储格式,能够更好地管理和利用LinkedIn的数据湖中的数据,提高数据
处理效率和可靠性。DataHub在LinkedIn内部得到了广泛应用,并已经开源。

(3)Comcast的数据湖平台

Comcast公司是美国最大的有线电视和互联网服务提供商之一,其数据湖平台是基于Iceberg的
设计与实现。该平台主要用于存储和管理各种类型的数据,包括用户数据、设备数据、运营数据等。
通过使用Iceberg格式作为数据存储格式,Comcast可以实现数据的版本控制和数据快照功能,方便对数据进行管理和分析。
Comcast的数据湖平台是其数据驱动战略的重要组成部分,为企业提供了更加灵活、可靠和高效的数据管理和分析解决方案,有助于推动其业务创新和发展。

\subsection{国内发展现状}

(1)腾讯的数据湖平台

腾讯数据湖平台是基于Iceberg存储格式的一款企业级数据存储和处理平台。在腾讯的数据湖平台中,
Iceberg扮演了重要的角色,被用于数据的存储和管理。腾讯的数据湖平台采用了Iceberg格式进行
数据存储,支持数据版本控制和快速查询。同时,腾讯还开发了自己的元数据管理系统,用于管理和
维护Iceberg表的元数据信息,包括表结构、数据版本、快照等信息。元数据管理系统与Iceberg表
相结合,实现了更加灵活和可靠的数据管理。腾讯的数据湖平台除了Iceberg,还包括Flume、Kafka、
Hadoop、Spark等工具,用于数据采集、处理和分析。

(2)阿里巴巴的MaxCompute数据湖平台

MaxCompute是阿里巴巴开发的一款企业级数据处理和分析平台,支持PB级别的数据处理和分析。
MaxCompute基于Iceberg存储格式,能够更好地管理和利用阿里巴巴的数据湖中的数据,提高
数据处理效率和可靠性。MaxCompute已经成为了阿里巴巴内部数据处理和分析的重要工具。

(3)京东的数据湖平台

京东的数据湖平台是基于Iceberg存储格式的一款企业级数据存储和处理平台,支持PB级别的
数据处理和分析。京东的数据湖平台具有更高的灵活性、可扩展性、查询性能和可靠性,支持
复杂的数据类型和多版本数据管理。京东的数据湖平台已经成为了京东内部数据处理和分析的重要工具。

(4)百度的数据湖平台

百度数据湖平台采用Iceberg格式作为存储格式,支持Apache Spark、Presto等分析工具。
在数据湖平台中,百度使用了Iceberg的版本管理和快照功能,支持数据版本控制和数据快照。
同时,百度也开发了自己的元数据管理系统,用于管理和维护Iceberg表的元数据信息,包括表
结构、数据版本、快照等信息。通过元数据管理系统,百度实现了更加灵活和可靠的数据管理。
除了Iceberg,百度的数据湖平台还包括Hadoop、Spark等工具,用于数据处理和分析。
在数据采集方面,百度也开发了自己的数据采集工具,包括数据采集、清洗、转换等功能。
在数据可视化和应用方面,百度采用自己开发的数据可视化工具,包括数据报表、图表等,方便用户进行数据分析和应用。

总体而言,这些企业应用Iceberg的案例表明,Iceberg作为一种新型的数据湖存储格式,
具有更高的灵活性、可扩展性、查询性能和可靠性。同时,Iceberg的广泛应用也证明了
其在实际应用中的可行性和有效性,为企业提供了更高效、更可靠、更灵活的数据存储和处理方案。

\section{本文的研究目标和内容}

该项目旨在解决以往数仓无法避免的架构延迟、高昂的数据修改成本等业务疼点,以实现实时接入、批流一体化、秒级分析、支持事务语义为目标。
为了实现该目标,本文设计并实现了基于Iceberg的数据湖分析系统,具体内容如下:

\subsection{数据源管理模块}

数据源管理模块主要对入湖任务的数据源进行管理,数据源是入湖任务必须设置的,是Iceberg入湖的源头,从数据源分类上来看,
数据湖分析系统支持关系型数据库源(mysql)以及消息队列数据源(tube、kafka、pulsar),该模块支持数据源创建、查看、编辑、删除功能。

\subsection{元数据管理模块}

元数据(Metadata)是描述其它数据的数据(data about other data)[15],或者说是用于提供某种资源的有关信息的
结构数据(structured data)。元数据是描述信息资源或数据等对象的数据,其使用目的在于:识别资源;评价资源;追踪
资源在使用过程中的变化;实现简单高效地管理大量网络化数据;实现信息资源的有效发现、查找、一体化组织和对使用资源的有效管理。
我们使用的元数据服务是内部部署的hive metastore,使用hive metastore来存储iceberg表的元数据,iceberg表即目标表,
是入湖任务的前提要求,可以创建新表或者关联已有的表,在创建入湖任务时即可选择对应的目标表。元数据管理模块主要对iceberg
元数据进行管理,包含两个主要的功能,分别是数据优化、表的创建与编辑,具体如下:

(1)数据优化

数据优化功能的目的是降低用户的运维成本,使用户可以一键启动该服务,不需要自己写java程序来优化,并会根据表的若干指标及历史执行情况
判断所需资源,并配有告警机制,及时通知专业运维处理。Iceberg表数据优化功能包括:合并小文件、清理过期快照数据、删除孤儿文件、生命周期管理。
合并小文件是为了减少小文件过多导致数据查询慢的问题,对于小文件合并我们在iceberg侧做了很大的优化,实现了计算资源的合理使用;
清理过期快照数据是清理commit过期的快照(snapshot),过期时间可以设置,以小时为粒度;
删除孤儿文件是清除commit因为冲突等原因产生的孤儿文件;
生命周期管理是对数据进行生命周期管理,需要所创建的表中带有Date或TimeStamp类型的时间字段,并对此字段进行操作删除过期数据;

(2)表的创建与编辑

表的创建既可以在数据湖分析系统上创建新的iceberg表,也可以关联在其它平台上已创建的iceberg表;
表的编辑可以将创建过的表进行修改编辑,支持字段的增加及删除操作。

\subsection{入湖任务管理模块}

数据入湖功能模块是DLA系统核心流程功能,目的是用户通过该功能将源数据表流程化入湖和查看已申请入湖任务执行情况。
入湖任务分为三类,分别为实时数据入湖、存量数据入湖和关系型数据入湖,三类入湖任务的创建都需要填写基础信息、源表、
目标表、参数及资源。

\subsection{数据探索模块}

数据探索功能是数据入湖后,用户需要进行数据查看或者数据分析时使用的工具,目前主要依赖内部查询平台实现数据探索功能。
我们的数据湖分析平台和查询平台之间是通过hive metastore(元数据服务)进行打通的,在查询平台上,用户可以编写sql进行数据查询,
底层支持的计算引擎有presto和spark。其中presto查询速度快,但是功能少,只支持查询功能;
spark虽然查询速度慢,但支持的语句功能比较多,支持DDL、DML、DQL等语句,当然查询平台也提供了api接口,通过这些
接口可以满足各种下游业务需求,如实时查询、BI报表、机器学习、交互式查询等。

\section{本文的组织结构}

第一章——介绍整个项目的背景及意义,分析数据湖分析系统领域的国内外发展现状,明确本研究的主要内容,
介绍本人的主要工作和论文的组织结构。

第二章——系统相关技术简介,本章主要阐述整个系统设计与实现过程中涉及到的主要技术,例如Iceberg、Flink、SpringBoot、Mysgl等等。

第三章——系统需求分析。本章节首先阐述了系统需求架构总览,接着对系统
的功能性需求进行细致的分析。得出解决问题的方案最后对系统的非功能性需求进行分析并进行总结。

第四章——系统概要设计。本章主要对系统的概要设计和详细设计进行阐述。包括系统总体功能模块、系统技术实现架构以及数据库详细设计。

第五章——系统详细设计与实现。本章节在第四章系统概要设计的基础上分
别对本系统的数据源管理模块、元数据管理模块、入湖任务管理模块、数据探索模块的实现进行阐述。

第六章——系统测试。本章节为整个系统设计测试用例,分别进行功能性测试、非功能性测试以及性能测试。对部分测试结果进行展示,最后得出测试结论。

第七章——结论与展望。本章节对系统开发工作做了整体的归纳和总结,并根据归纳和总结对本系统未来的工作内容进行展望。

\section{本章小结}

本章首先介绍了基于Iceberg的数据湖分析系统的背景、意义以及国内外的相关研究现状,
然后确立项目的研究目标,并从元数据管理和数据使用两方面阐述项目研究内容,最后对论文的组织结构进行说明。
