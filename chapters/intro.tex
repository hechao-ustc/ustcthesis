% !TeX root = ../main.tex

\chapter{绪论}

本章节主要对论文研究的项目--基于Iceberg的数据湖分析系统的设计与
实现,进行概要介绍。分别从课题背景和意义、国内外研究现状进行分析,制定了
本论文的研究目标,细化论文的研究内容,最后概述本论文的组织结构。

\section{课题背景和意义}

如今,随着互联网的发展和层出不穷的各种应用,互联网产生着大量的数据,
如何有效存储和处理这些大规模数据成为了一个亟待解决的问题[1]。为此,
数据仓库已是企业内部进行数据洞察的标准服务。但是传统基于Hadoop技术[2]
的数据仓库从数据导入到数据分析每个环节都有较大的延迟,使得数据分析的时效
性大大变低;同时对于数据分析场景的拓展,传统单一的数仓架构也没法满足多变
的数据分析需求。现如今Hadoop生态圈的数据仓库大多面向的是离线场景,数据
从接入直到展示存在着小时级以上的延迟,这其中一方面是因为批处理框架本身延
迟的限制,另一方面也是因为对于变化数据捕捉能力的缺失。

为了解决现有数仓架构的时延问题,业界也在该领域进行了较多的探索,其中最主要的
探索是利用Lambda架构[3]解决海量数据处理的时延问题。Lambda架构是一种常见的
数据处理体系结构,它的数据处理依赖流式计算层(Streaming Layer)和批处理
计算层(Batch Layer)的双重计算。Lambda架构虽然解决了海量数据处理的时延问题,
但是“流批分离”的处理链路增大了研发的复杂性。为了解决Lambda架构的复杂性,也出现
有了诸多的改进方案,比如Kappa架构[4]。Kappa架构充分利用了流计算天然的分布式特征,
注定了他的扩展性更好。通过加大流计算的并发性,加大流式数据的“时间窗口”,来统一批处理
与流式处理两种计算模式。这样的架构简单,避免了维护两套系统还需要保持结果一致的问题,
也很好解决了数据订正问题,但它也有它的问题:消息中间件回放困难,修改缺乏灵活性,同时
OLAP分析性能低下,无法利用到列裁剪、谓词下推、向量化等现代引擎的常用优化手段。

为了解决上述Lambda架构和Kappa架构引入的问题,我们引入了数据湖技术。如果把数据比作
大自然的水,那么各个江川河流的水未经加工,源源不断地汇聚到数据湖中[5,6,7,8]。数据湖
是一个以原始格式存储数据的存储库或系统。它按原样存储数据,而无需事先对数据进行结构化处
理。而本文中的数据湖是基于Apache Iceberg实现的,根据官方的定义,Iceberg是一种表格
式(table format)。我们可以简单理解为它是基于计算层(flink、spark)和存储层
(orc[9]、parquet[10])的一个中间层,它与底层的存储格式最大的区别是,它并不定义数据
存储方式,而是定义了数据、元数据的组织方式,向上提供统一的“表”的语义。在hive建立一个
Iceberg格式的表。用flink或者spark写入iceberg,然后再通过其他方式来读取这个表,比如
spark、flink、presto等[16]。我们基于Iceberg并依托公司内部实时流计算平台和已有的交
互式查询引擎Presto和Spark的能力,打造了下一代的实时数仓-数据湖分析系统DLA。DLA可以轻
松完成T+0实时入湖,并支持批流融合、秒级分析、事务语义、挖掘和探索数据价值等。

\section{国内外研究现状}

自2011年“数据湖”概念被提出,数据湖架构一直在不断变革和发展,很多厂商都提出了不同的解决方案。
华为云数据湖解决方案基于先进的云上系统架构,提供数据湖查询分析服务Data Lake Insight,简称DLI。
DLI作为基于Spark的华为云数据湖探索服务,解决了用户使用大数据时存在的以下痛点:

(1)传统的使用大数据需要繁琐的步骤,需要人工配置可用的物理环境,包括机房、网络、集群等的配置,
然后手动安装大数据相关软件,参与日常的维护升级工作,这严重依赖平台技术,有较高的门槛。另外随着
数据量的增加,对现有集群难以进行扩容。DLI是完全托管的,用户无需维护任何的基础设施,无需做任何ETL,
就可以使用标准 SOL进行数据查询分析,即来即用,这为企业节约了大量的人力时间成本。

(2)原始数据类型多样、格式不一,比如CSV、TXT、JSON、Parquet、ORC等格式,另外数据来源也是多种多样,
包括关系型Mysql,Kafka,Hbase等数据来源。DUI完全兼容Spark生态,天然支持多种数据源和数据格式。
另外基于华为云的特色扩展数据源,支持跨源联邦查询能力,使用户可以直接使用sql实现对云上多数据源探索。
例如使用DLI直接访问OBS、Cloud Table,基干AI技术,通过OCR对图片数据进行处理和分析。

(3)企业面临不同部门或用户之间数据难以共享的问题。DLI企业级多租户管理,用户可以选择对资源进行配置,
包括 CPU,内存,不同用户对应不同资源队列,保障作业相互不受影响,同时用户可以选择表或者部分列进行权
限设置,以保障企业内部各部门之间的数据安全访问,实现数据精细化管控,保障不同粒度的权限控制。通过元数据
层面实现,不涉及数据层面的操作。

(4)企业业务系统需要面对多种业务应用需求,比如多数据源的联合查询、BI分析、机器学习等需求。通过跨源数据
联合查询,无需加载免搬迁、不用 ETL、使用标准SQL就可以对云上异构数据源做联合查询分析。交互式多维分析,
通过丰富的报表组件,支持实时定时任务,实现TB级数据秒级响应。

此外,DLI采用业界先进的CarbonData[11]存储技术,结合分区表,预聚合,缓存加速,索引下压等技术加速查询
性能联合华为OBS DFV技术,采用索引下压等技术深度优化查询性能。

数据湖相关技术发展迅速,除了华为的数据湖探索服务,国内外很多产商都开发了数据湖产品,比如阿里云数据湖分析(DLA)、
金山数据湖分析(KDLA)、AWS数据湖[14]等。其中AWS作为知名的大数据和云计算服务以及云解决方案提供商,
是最早参与数据湖研究的公司之一,目前已经形成了一套完整的技术方案。AWS数据湖为用户提供了一系列的管理工具,
与S3作为核心存储的数据湖紧密集成,提供企业级整体方案。此外,用户可以利用Glue Catalog、Amazon ES等工具来进行元数据管理[15]。

\section{本文的研究目标和内容}

该项目旨在解决以往数仓无法避免的架构延迟、高昂的数据修改成本等业务疼点,以实现实时接入、批流一体化、秒级分析、支持事务语义为目标。
为了实现该目标,本文设计并实现了基于Iceberg的数据湖分析系统,具体内容如下:

\subsection{数据源管理模块}

数据源管理模块主要对入湖任务的数据源进行管理,数据源是入湖任务必须设置的,是Iceberg入湖的源头,从数据源分类上来看,
数据湖分析系统支持关系型数据库源(mysql)以及消息队列数据源(tube、kafka、pulsar),该模块支持数据源创建、查看、编辑、删除功能。

\subsection{元数据管理模块}

元数据(Metadata)是描述其它数据的数据(data about other data)[15],或者说是用于提供某种资源的有关信息的
结构数据(structured data)。元数据是描述信息资源或数据等对象的数据,其使用目的在于:识别资源;评价资源;追踪
资源在使用过程中的变化;实现简单高效地管理大量网络化数据;实现信息资源的有效发现、查找、一体化组织和对使用资源的有效管理。
我们使用的元数据服务是内部部署的hive metastore,使用hive metastore来存储iceberg表的元数据,iceberg表即目标表,
是入湖任务的前提要求,可以创建新表或者关联已有的表,在创建入湖任务时即可选择对应的目标表。元数据管理模块主要对iceberg
元数据进行管理,包含两个主要的功能,分别是数据优化、表的创建与编辑,具体如下:

\subsubsection{数据优化}

数据优化功能的目的是降低用户的运维成本,使用户可以一键启动该服务,不需要自己写java程序来优化,并会根据表的若干指标及历史执行情况
判断所需资源,并配有告警机制,及时通知专业运维处理。Iceberg表数据优化功能包括:合并小文件、清理过期快照数据、删除孤儿文件、生命周期管理。
合并小文件是为了减少小文件过多导致数据查询慢的问题,对于小文件合并我们在iceberg侧做了很大的优化,实现了计算资源的合理使用;
清理过期快照数据是清理commit过期的快照(snapshot),过期时间可以设置,以小时为粒度;
删除孤儿文件是清除commit因为冲突等原因产生的孤儿文件;
生命周期管理是对数据进行生命周期管理,需要所创建的表中带有Date或TimeStamp类型的时间字段,并对此字段进行操作删除过期数据;

\subsubsection{表的创建与编辑}

表的创建既可以在数据湖分析系统上创建新的iceberg表,也可以关联在其它平台上已创建的iceberg表;
表的编辑可以将创建过的表进行修改编辑,支持字段的增加及删除操作。

\subsection{入湖任务管理模块}

数据入湖功能模块是DLA系统核心流程功能,目的是用户通过该功能将源数据表流程化入湖和查看已申请入湖任务执行情况。
入湖任务分为三类,分别为实时数据入湖、存量数据入湖和关系型数据入湖,三类入湖任务的创建都需要填写基础信息、源表、
目标表、参数及资源。

\subsection{数据探索模块}

数据探索功能是数据入湖后,用户需要进行数据查看或者数据分析时使用的工具,目前主要依赖内部查询平台实现数据探索功能。
我们的数据湖分析平台和查询平台之间是通过hive metastore(元数据服务)进行打通的,在查询平台上,用户可以编写sql进行数据查询,
底层支持的计算引擎有presto和spark。其中presto查询速度快,但是功能少,只支持查询功能;
spark虽然查询速度慢,但支持的语句功能比较多,支持DDL、DML、DQL等语句,当然查询平台也提供了api接口,通过这些
接口可以满足各种下游业务需求,如实时查询、BI报表、机器学习、交互式查询等。

\section{本文的组织结构}

第一章——介绍整个项目的背景及意义,分析数据湖分析系统领域的国内外发展现状,明确本研究的主要内容,
介绍本人的主要工作和论文的组织结构。

第二章——系统相关技术简介,本章主要阐述整个系统设计与实现过程中涉及到的主要技术,例如Iceberg、Flink、SpringBoot、Mysgl等等。

第三章——系统需求分析。本章节首先阐述了系统需求架构总览,接着对系统
的功能性需求进行细致的分析。得出解决问题的方案最后对系统的非功能性需求进行分析并进行总结。

第四章——系统概要设计。本章主要对系统的概要设计和详细设计进行阐述。包括系统总体功能模块、系统技术实现架构以及数据库详细设计。

第五章——系统详细设计与实现。本章节在第四章系统概要设计的基础上分
别对本系统的数据源管理模块、元数据管理模块、入湖任务管理模块、数据探索模块的实现进行阐述。

第六章——系统测试。本章节为整个系统设计测试用例,分别进行功能性测试、非功能性测试以及性能测试。对部分测试结果进行展示,最后得出测试结论。

第七章——结论与展望。本章节对系统开发工作做了整体的归纳和总结,并根据归纳和总结对本系统未来的工作内容进行展望。

\section{本章小结}

本章首先介绍了基于Iceberg的数据湖分析系统的背景、意义以及国内外的相关研究现状,
然后确立项目的研究目标,并从元数据管理和数据使用两方面阐述项目研究内容,最后对论文的组织结构进行说明。


\subsection{二级节标题}

\subsubsection{三级节标题}

\paragraph{四级节标题}

\subparagraph{五级节标题}

本模板 \pkg{ustcthesis} 是中国科学技术大学本科生和研究生学位论文的 \LaTeX{}
模板, 按照《\href{https://gradschool.ustc.edu.cn/static/upload/article/picture/ce3b02e5f0274c90b9331ef50ae1ac26.pdf}
{中国科学技术大学研究生学位论文撰写手册}》(以下简称《撰写手册》)和
《\href{https://www.teach.ustc.edu.cn/?attachment_id=13867}
{中国科学技术大学本科毕业论文(设计)格式}》的要求编写。

Lorem ipsum dolor sit amet, consectetur adipiscing elit, sed do eiusmod tempor
incididunt ut labore et dolore magna aliqua.
Ut enim ad minim veniam, quis nostrud exercitation ullamco laboris nisi ut
aliquip ex ea commodo consequat.
Duis aute irure dolor in reprehenderit in voluptate velit esse cillum dolore eu
fugiat nulla pariatur.
Excepteur sint occaecat cupidatat non proident, sunt in culpa qui officia
deserunt mollit anim id est laborum.



\section{脚注}

Lorem ipsum dolor sit amet, consectetur adipiscing elit, sed do eiusmod tempor
incididunt ut labore et dolore magna aliqua.
\footnote{Ut enim ad minim veniam, quis nostrud exercitation ullamco laboris
  nisi ut aliquip ex ea commodo consequat.
  Duis aute irure dolor in reprehenderit in voluptate velit esse cillum dolore
  eu fugiat nulla pariatur.}
